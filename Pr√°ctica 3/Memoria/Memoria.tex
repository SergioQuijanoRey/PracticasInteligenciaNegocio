\documentclass[11pt]{article}

% Paquetes
%===================================================================================================

% Establecemos los márgenes
\usepackage[a4paper, margin=1in]{geometry}

% Separacion entre parrafos
\setlength{\parskip}{1em}

% Paquete para incluir codigo
\usepackage{listings}

% Paquete para incluir imagenes
\usepackage{graphicx}
\graphicspath{ {./images/} }

% Para fijar las imagenes en la posicion deseada
\usepackage{float}

% Para que el codigo acepte caracteres en utf8
\lstset{literate=
  {á}{{\'a}}1 {é}{{\'e}}1 {í}{{\'i}}1 {ó}{{\'o}}1 {ú}{{\'u}}1
  {Á}{{\'A}}1 {É}{{\'E}}1 {Í}{{\'I}}1 {Ó}{{\'O}}1 {Ú}{{\'U}}1
  {à}{{\`a}}1 {è}{{\`e}}1 {ì}{{\`i}}1 {ò}{{\`o}}1 {ù}{{\`u}}1
  {À}{{\`A}}1 {È}{{\'E}}1 {Ì}{{\`I}}1 {Ò}{{\`O}}1 {Ù}{{\`U}}1
  {ä}{{\"a}}1 {ë}{{\"e}}1 {ï}{{\"i}}1 {ö}{{\"o}}1 {ü}{{\"u}}1
  {Ä}{{\"A}}1 {Ë}{{\"E}}1 {Ï}{{\"I}}1 {Ö}{{\"O}}1 {Ü}{{\"U}}1
  {â}{{\^a}}1 {ê}{{\^e}}1 {î}{{\^i}}1 {ô}{{\^o}}1 {û}{{\^u}}1
  {Â}{{\^A}}1 {Ê}{{\^E}}1 {Î}{{\^I}}1 {Ô}{{\^O}}1 {Û}{{\^U}}1
  {ã}{{\~a}}1 {ẽ}{{\~e}}1 {ĩ}{{\~i}}1 {õ}{{\~o}}1 {ũ}{{\~u}}1
  {Ã}{{\~A}}1 {Ẽ}{{\~E}}1 {Ĩ}{{\~I}}1 {Õ}{{\~O}}1 {Ũ}{{\~U}}1
  {œ}{{\oe}}1 {Œ}{{\OE}}1 {æ}{{\ae}}1 {Æ}{{\AE}}1 {ß}{{\ss}}1
  {ű}{{\H{u}}}1 {Ű}{{\H{U}}}1 {ő}{{\H{o}}}1 {Ő}{{\H{O}}}1
  {ç}{{\c c}}1 {Ç}{{\c C}}1 {ø}{{\o}}1 {å}{{\r a}}1 {Å}{{\r A}}1
  {€}{{\euro}}1 {£}{{\pounds}}1 {«}{{\guillemotleft}}1
  {»}{{\guillemotright}}1 {ñ}{{\~n}}1 {Ñ}{{\~N}}1 {¿}{{?`}}1 {¡}{{!`}}1
}

% Para que no se salgan las lineas de codigo
% Para fijar una fuente que resalte
\lstset{breaklines=true, basicstyle=\ttfamily}

% Para que los metadatos que escribe latex esten en español
\usepackage[spanish]{babel}
\decimalpoint % Para que no se cambie el punto a la coma

% Para la bibliografia
% Sin esto, los enlaces de la bibliografia dan un error de compilacion
\usepackage{url}

% Para que se puedan clicar los enlaces
\usepackage{hyperref}

% Para mostrar graficas de dos imagenes, cada una con su caption, y con un caption comun
\usepackage{subcaption}

% Simbolo de los numeros reales
\usepackage{amssymb}

% Para que los codigos tengan una fuente distinta
\usepackage{courier}

\lstdefinestyle{CustomStyle}{
  language=Python,
  numbers=left,
  stepnumber=1,
  numbersep=10pt,
  tabsize=4,
  showspaces=false,
  showstringspaces=false
  basicstyle=\tiny\ttfamily,
}

% Para referenciar secciones usando el nombre de las secciones
\usepackage{nameref}

% Para enumerados dentro de enumerados
\usepackage{enumitem}

% Para mejores tablas
\usepackage{tabularx}

% Para poder tener el mismo identificador en dos tablas separadas
\usepackage{caption}

% Mostrar la página de las referencias en el indice del documento
\usepackage[nottoc,numbib]{tocbibind}

% Para mostrar las matrices
\usepackage{amsmath}

% Para que las notas al pie de pagina queden bien abajo
\usepackage[bottom]{footmisc}

% Comandos personalizados
%===================================================================================================

% Para realizar las citas de forma corta
\newcommand{\customcite}[1]{\emph{"\ref{#1}. \nameref{#1}"}}

% Para entrecomillar un texto
\newcommand{\entrecomillado}[1]{\emph{``#1''}}

% Metadatos del documento
%===================================================================================================
\title{
    {Inteligencia de Negocio - Práctica 2} \\
    {Análisis Relacional mediante Segmentación}
}

\author{
    {Sergio Quijano Rey - 72103503k}\\
    {sergioquijano@correo.ugr.es} \\
    {5º Doble Grado Ingeniería Informática y Matemáticas} \\
    {Grupo de prácticas 1}\\
}

\date{\today}

% Separacion entre parrafos
\setlength{\parskip}{1em}

% Contenido del documento
%===================================================================================================
\begin{document}

% Portada del documento
\maketitle
\pagebreak

% Indice de contenidos
\tableofcontents

% Lista de figuras
% Uso el addtocontents para que no se muestre la seccion de indice de figuras en el indice inicial

\addtocontents{toc}{\setcounter{tocdepth}{-10}}
\listoffigures

% TODO -- no tenemos cuadros en esta memoria
\listoftables

% TODO -- tampoco tenemos codigos de relevancia
% \lstlistoflistings
\addtocontents{toc}{\setcounter{tocdepth}{3}}

\pagebreak

\section{Tabla submissions}

A continuación mostramos la tabla de \emph{submissions} en la que queda reflejada el esfuerzo por mejorar los resultados:

% TODO -- es muy importante que incluyamos la imagen con la tabla de todas las submissions

En la figura anterior podemos ver que las dos primeras entregas no tuvieron \emph{score}. Esto se debe a que subimos los resultados con un formato erróneo.

\pagebreak

\section{Introducción}

\subsection{Descripción del problema}

Según la descripción del problema de \emph{Driven Data} \cite{driven_data_problem_description:online}, debemos calcular la probabilidad de que una persona se vacune de dos tipos de vacunas distintas. Es decir, deberemos calcular dos probabilidades, una por cada tipo de vacuna con la que se trabaja. Las vacunas son para el virus \emph{h1n1} y para la gripe.

Disponemos de 39 columnas. Una de ellas es para el identificador de la persona encuestada (que no usaremos) y dos de ellas son las etiquetas a predecir. Por tanto disponemos de 36 columnas para llevar a cabo la tarea de aprendizaje. Tenemos columnas de tipo numérico y de tipo categórico. Por otro lado, tenemos 26707 ejemplos en nuestra base de datos de entrenamiento.

La métrica a optimizar, y en la que se basará el \emph{ranking} de la competición, será el área bajo la curva ROC. Como tenemos dos etiquetas a predecir, será la media del área bajo la curva ROC para las dos etiquetas por separado.

Por tanto, las propuestas subidas a la plataforma \emph{Driven Data} deberán ser dos valores probabilísticos entre 0 y 1, y no simplemente valores de clasificación binaria $\{0, 1\}$.

Respecto a la plataforma donde se desarrolla la competición, dejan a nuestra disposición dos conjuntos de datos. El conjunto de entrenamiento, etiquetado, y el conjunto de test, sin etiquetar. A partir del conjunto de test deberemos generar la propuesta que subimos a la plataforma. Además, solo dispondremos de 3 propuestas al día. Cuando se realiza una propuesta, se conoce el \emph{score} en test de forma inmediata.

\subsection{Consideraciones iniciales}

\pagebreak

\section{Desarrollo de los experimentos}

\subsection{Resumen de los experimentos realizados}

\pagebreak

\subsection{Experimento 1}

\pagebreak

% Bibliografia
\bibliography{./References}
\bibliographystyle{ieeetr}

\end{document}
