\documentclass[11pt]{article}

% Paquetes
%===================================================================================================

% Establecemos los márgenes
\usepackage[a4paper, margin=1in]{geometry}

% Separacion entre parrafos
\setlength{\parskip}{1em}

% Paquete para incluir codigo
\usepackage{listings}

% Paquete para incluir imagenes
\usepackage{graphicx}
\graphicspath{ {./Imagenes/} }

% Para fijar las imagenes en la posicion deseada
\usepackage{float}

% Para que el codigo acepte caracteres en utf8
\lstset{literate=
  {á}{{\'a}}1 {é}{{\'e}}1 {í}{{\'i}}1 {ó}{{\'o}}1 {ú}{{\'u}}1
  {Á}{{\'A}}1 {É}{{\'E}}1 {Í}{{\'I}}1 {Ó}{{\'O}}1 {Ú}{{\'U}}1
  {à}{{\`a}}1 {è}{{\`e}}1 {ì}{{\`i}}1 {ò}{{\`o}}1 {ù}{{\`u}}1
  {À}{{\`A}}1 {È}{{\'E}}1 {Ì}{{\`I}}1 {Ò}{{\`O}}1 {Ù}{{\`U}}1
  {ä}{{\"a}}1 {ë}{{\"e}}1 {ï}{{\"i}}1 {ö}{{\"o}}1 {ü}{{\"u}}1
  {Ä}{{\"A}}1 {Ë}{{\"E}}1 {Ï}{{\"I}}1 {Ö}{{\"O}}1 {Ü}{{\"U}}1
  {â}{{\^a}}1 {ê}{{\^e}}1 {î}{{\^i}}1 {ô}{{\^o}}1 {û}{{\^u}}1
  {Â}{{\^A}}1 {Ê}{{\^E}}1 {Î}{{\^I}}1 {Ô}{{\^O}}1 {Û}{{\^U}}1
  {ã}{{\~a}}1 {ẽ}{{\~e}}1 {ĩ}{{\~i}}1 {õ}{{\~o}}1 {ũ}{{\~u}}1
  {Ã}{{\~A}}1 {Ẽ}{{\~E}}1 {Ĩ}{{\~I}}1 {Õ}{{\~O}}1 {Ũ}{{\~U}}1
  {œ}{{\oe}}1 {Œ}{{\OE}}1 {æ}{{\ae}}1 {Æ}{{\AE}}1 {ß}{{\ss}}1
  {ű}{{\H{u}}}1 {Ű}{{\H{U}}}1 {ő}{{\H{o}}}1 {Ő}{{\H{O}}}1
  {ç}{{\c c}}1 {Ç}{{\c C}}1 {ø}{{\o}}1 {å}{{\r a}}1 {Å}{{\r A}}1
  {€}{{\euro}}1 {£}{{\pounds}}1 {«}{{\guillemotleft}}1
  {»}{{\guillemotright}}1 {ñ}{{\~n}}1 {Ñ}{{\~N}}1 {¿}{{?`}}1 {¡}{{!`}}1
}

% Para que no se salgan las lineas de codigo
% Para fijar una fuente que resalte
\lstset{breaklines=true, basicstyle=\ttfamily}

% Para que los metadatos que escribe latex esten en español
\usepackage[spanish]{babel}
\decimalpoint % Para que no se cambie el punto a la coma

% Para la bibliografia
% Sin esto, los enlaces de la bibliografia dan un error de compilacion
\usepackage{url}

% Para mostrar graficas de dos imagenes, cada una con su caption, y con un caption comun
\usepackage{subcaption}

% Simbolo de los numeros reales
\usepackage{amssymb}

% Para que los codigos tengan una fuente distinta
\usepackage{courier}

\lstdefinestyle{CustomStyle}{
  language=Python,
  numbers=left,
  stepnumber=1,
  numbersep=10pt,
  tabsize=4,
  showspaces=false,
  showstringspaces=false
  basicstyle=\tiny\ttfamily,
}

% Para referenciar secciones usando el nombre de las secciones
\usepackage{nameref}

% Para enumerados dentro de enumerados
\usepackage{enumitem}

% Para mejores tablas
\usepackage{tabularx}

% Para poder tener el mismo identificador en dos tablas separadas
\usepackage{caption}

% Mostrar la página de las referencias en el indice del documento
\usepackage[nottoc,numbib]{tocbibind}

% Para mostrar las matrices
\usepackage{amsmath}

% Metadatos del documento
%===================================================================================================
\title{
    {Prácticas Inteligencia de Negocio}
}

\author{
    {Sergio Quijano Rey - 72103503k}\\
    {sergioquijano@correo.ugr.es} \\
    {5º Doble Grado Ingeniería Informática y Matemáticas}
}

\date{\today}

% Separacion entre parrafos
\setlength{\parskip}{1em}

% Contenido del documento
%===================================================================================================
\begin{document}

% Portada del documento
\maketitle
\pagebreak

% Indice de contenidos
\tableofcontents

% Lista de figuras
% Uso el addtocontents para que no se muestre la seccion de indice de figuras en el indice inicial

\addtocontents{toc}{\setcounter{tocdepth}{-10}}
\listoffigures
\addtocontents{toc}{\setcounter{tocdepth}{2}}

\pagebreak

\section{Introducción}

En esta sección hablaremos de cada uno de los problemas abordados, tratando las particularidades de cada caso y algunas consideraciones en base a estas peculiaridades tratadas.

\subsection{Información general}

En todas las partes en las que necesitemos usar números aleatorios, usaremos la semilla 123456789 para poder reproducir bajo las mismas condiciones los experimentos y para que las comparaciones entre los distintos algoritmos sean lo más justas posible.

\subsection{\emph{Heart Failure Prediction}}

% TODO -- los enlaces deben ser clicables
En primer lugar, la fuente original del \emph{dataset} se puede encontrar en \url{https://www.kaggle.com/fedesoriano/heart-failure-prediction}. Aunque podemos estar trabajando con un \emph{dataset} ligeramente modificado por los profesores de la asignatura.

Analizando los datos con \lstinline{KNIME}, como explicaremos más adelante, vemos que tenemos 12 columnas (una por cada una de las características del \emph{datasets}). Además, no tenemos \emph{missing values}.


\pagebreak

% Bibliografia
\bibliography{./References}
\bibliographystyle{ieeetr}

\end{document}
